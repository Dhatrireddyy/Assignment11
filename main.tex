\documentclass{beamer}
\usetheme{CambridgeUS}

\title{Assignment 11: Papoulis Chapter 7 }
\author{Dhatri Reddy}
\date{\today}
\logo{\large \LaTeX{}}

\usepackage{amsmath}
\usepackage{romannum}
\usepackage{enumitem}
\setbeamertemplate{caption}[numbered]{}
\providecommand{\pr}[1]{\ensuremath{\Pr\left(#1\right)}}
\providecommand{\cbrak}[1]{\ensuremath{\left\{#1\right\}}}
\providecommand{\brak}[1]{\ensuremath{\left(#1\right)}}


\begin{document}

\begin{frame}
    \titlepage 
\end{frame}

\logo{}

\begin{frame}{Outline}
    \tableofcontents
\end{frame}

\section{Question}
\begin{frame}{Question}
    \begin{block}{Problem 7.24}
        Show that if the random variables $x_{i}$ are of continuous type and independent, then, for sufficiently large n, the density of $\sin{\cbrak{x_{1} + ... + x_{n}}}$ is nearly equal to density of $\sin{x}$, where x is a random variable uniform in the interval \brak{-\pi, \pi}
\end{block}
\end{frame}

\section{Solution}
\begin{frame}
\frametitle{Solution}

The density $f_{z}\brak{z}$ of the sum $z = x_{1} + ... + x_{n}$ tends to a normal curve with variance $\sigma_{1}^{2} + ... + \sigma_{n}^{2}\to \infty$ as $n \to \infty$.

x is uniform in the interval \brak{-\pi, \pi} and $y = \sin{x}$. 
In this case 

\end{frame}

\section{Solving}
\begin{frame}
\frametitle{Solving}
\begin{align}
    \phi_{y}\brak{\omega} &= \int_{-\infty}^\infty e^{j\omega\sin{x}}f\brak{x}dx\\
    &= \frac{1}{2\pi}\int_{-\pi}^\pi e^{j\omega\sin{x}}dx\\
    &= \frac{1}{2\pi}\int_{-\pi}^\frac{-\pi}{2} e^{j\omega\sin{x}} + \int_{\frac{-\pi}{2}}^\frac{\pi}{2} e^{j\omega\sin{x}} dx + \int_{\frac{\pi}{2}}^{0} e^{j\omega\sin{x}} dx\\
    dy &= \cos{x} dx = \brak{1-y^{2}}^{\frac{1}{2}} dx\\
    \phi_{y}\brak{\omega} &= \frac{1}{2\pi}\int_{0}^{-1}e^{j\omega}\brak{1-y^{2}}^{-\frac{1}{2}} dy  + \int_{-1}^{1}e^{j\omega}\brak{1-y^{2}}^{-\frac{1}{2}} dy\\
    & +\int_{1}^{0}e^{j\omega}\brak{1-y^{2}}^{-\frac{1}{2}} dy
\end{align}
\end{frame}

\begin{frame}
\frametitle{Solution}
This leads to the conclusion that 
\begin{align}
    f_{y}\brak{y} &= \frac{1}{\pi \brak{1-y^2}^\frac{1}{2}}\\
\end{align}

Hence, $f_{z}\brak{z}$ tends to a constant in any interval of length of $2\pi$.
\end{frame}

\end{document}